\documentclass{beamer} 
\usepackage[utf8]{inputenc}
\usepackage{polski}

\usetheme{Copenhagen}

\author[]{Paweł Koniszewski \\ Przemysław Kubicki \\ Tomasz Wesołowski}

\institute[Politechnika Gdańska, ETI]{Politechnika Gdańska, Wydział Elektrotechniki, Telekomunikacji i Informatyki}
\date{Gdańsk, \today}
\title[Armadio]{Program rozszerzonej rzeczywistości do meblowania pomieszczeń \\ ARMADIO}

\newtheorem{dfn}{Definicja}
\begin{document}

\frame{\titlepage}

\frame{
	\frametitle{Agenda}
	\tableofcontents
}

\section{Rozszerzona rzeczywistość}
\subsection{Definicja}
\frame{
	\frametitle{Definicja}
	\begin{dfn}
	Rzeczywistość wirtualna stworzona z połączenia prawdziwego i wirtualnego świata
	\end{dfn}
	\begin{dfn}
Ronald Azuma zdefiniował rozszerzą rzeczywistość jako:
	\begin{itemize}
		\item łączącą w sobie świat realny oraz rzeczywistość wirtualną
		\item interaktywną w czasie rzeczywistym
		\item umożliwiącą swobodę ruchów w trzech wymiarach
	\end{itemize}
	\end{dfn}
}
	
\subsection{Zastosowania}
\frame{
	\frametitle{Niektóre z zastosowań}
	\begin{itemize}
		\item marketing
		\item edukacja
		\item medycyna
		\item lotnictwo
		\item \ldots
	\end{itemize}
}

\subsection{Przykłady}
\frame{
	\frametitle{Przykłady}
	\begin{itemize}
	\item Google glass
	\item Layar, Wikitude
	\item Filmy ("Raport mniejszości", "Ironman", \ldots)
	\item Gry ("ArDefender", \ldots)
	\end{itemize}
}

\section{Armadio}
\subsection{Geneza}
\frame{

}

\subsection{Istniejące rozwiązania - reaserch}
\frame{}
\subsection{Technologia}
\frame{
	\frametitle{Unity}
	\begin{dfn}
	$\mathbf{Unity}$ - zintegrowane narzędzie do tworzenia gier trójwymiarowych lub innych materiałów interaktywnych
	\end{dfn}
	$\mathbf{Cechy:}$
	\begin{itemize}
	\item fajne
	\item multiplatformowość
	\item multijęzykowość
	\item darmowe
	\end{itemize}
}
\subsection{Postęp prac}
\frame{
	\frametitle{Prezentacja naszej dotychczasowej pracy}
	voila!
}

\end{document}